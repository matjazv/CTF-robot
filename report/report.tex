
\documentclass[12pt,a4paper,openany]{book}

%Uporabljeni paketi
\usepackage{fancyhdr}
\usepackage{graphicx}
\usepackage{color}
\usepackage{xcolor}
\usepackage{listings}
\usepackage[slovene]{babel}

\usepackage[cp1250]{inputenc}
\usepackage[pdftex,bookmarks=true]{hyperref} %omogo?a zaznamke

%Velikost strani - dvostransko
\oddsidemargin 1.4cm
\evensidemargin 0.35cm
\textwidth 14cm
\topmargin 0.26cm
\headheight 0.6cm
\headsep 1.5cm
\textheight 20cm

%Nastavitev glave in repa strani
\pagestyle{fancy}
\fancyhead{}
\renewcommand{\chaptermark}[1]{\markboth{\textsf{Poglavje \thechapter:\ #1}}{}}
\renewcommand{\sectionmark}[1]{\markright{\textsf{\thesection\  #1}}{}}
\fancyhead[RE]{\leftmark}
\fancyhead[LO]{\rightmark}
\fancyhead[LE,RO]{\thepage}
\fancyfoot{}
\renewcommand{\headrulewidth}{0.0pt}
\renewcommand{\footrulewidth}{0.0pt}

%********************************************

\begin{document}

% stran 1 med uvodnimi listi
\thispagestyle{empty} 

\begin{center}
{\large 
UNIVERZA V LJUBLJANI\\
FAKULTETA ZA RAÈUNALNIŠTVO IN INFORMATIKO\\
}

 \includegraphics[scale=0.2,keepaspectratio=true]{./pictures/uni_logo.png}

\vspace{1.5cm}
{\LARGE Rok Kek}\\

\vspace{2cm}
\textsc{\textbf{\LARGE 
Testiranje stopnje asociativnosti predpomnilnika
in vpliv na hitrost izvajanje programov\\ 
}}

\vspace{2cm}
{ SEMINARSKA NALOGA}\\
{ NA UNIVERZITETNEM ŠTUDIJU }\\

\vspace{2cm} 
{\Large Mentor: prof. dr. Nikolaj Zimic}

\vfill
{\Large Ljubljana, 2010}
\end{center}

\newpage

%********************************************

\renewcommand\thepage{} 
\tableofcontents 
\renewcommand\thepage{\arabic{page}}

\thispagestyle{empty}


%********************************************

\chapter*{Seznam uporabljenih kratic in simbolov}

\thispagestyle{empty}

\textbf{LRU} (Least Recently Used): Menjalna strategija, pri kateri menjamo blok, ki ni bil v uporabi najdalj èasa.\\
\textbf{MRU} (Most Recently Used): Menjalna strategija, pri kateri menjamo blok, ki je bil nazadnje uporabljen.\\
\textbf{RR} (Random Replacement): Menjalna strategija, pri kateri nakljuèno izberemo blok ki ga bomo menjali.\\
\textbf{L1} : Predpomnilnik na prvem nivoju\\
\textbf{L2} : Predpomnilnik na drugem nivoju\\
\textbf{LFU} (Least-Frequently Usedt): Menjalna strategija, pri kateri menajmo blok, ki je bil najmanjkrat uporabljen.
%\cleardoublepage

\clearpage{\pagestyle{empty}\cleardoublepage}

%********************************************
%zacno se glavni listi, ki so numerirani z arabskimi stevilkami

\setcounter{page}{1}
\pagenumbering{arabic}

\chapter*{Povzetek}

\addcontentsline{toc}{chapter}{Povzetek}

V seminarski nalogi sem testiral stopnjo asociativnosti predpomnilnika oziroma njen vpliv na hitrost izvajanje programov. Teste sem
izvajal na prenosnem raèunalniku nižjega cenovnega razreda. Za testiranje sem napisal program, v programskem jeziku C in teste poganjal,
na Linux operaciskem sistemu. Programski jezik C sem si izbral zato, ker je dovolj nizkonivojski, da lahko izklapljamo prekinitve, in
omogoèimo pogoje za testiranje takih detailov procesorja. Linux operaciski sistem pa pa zato, ker je odprtokoden, in je zanj veè dokumentacije
o sami implementaciji sisteme ter knjižnic za delo s strojno opremo. Program sem sprva želel napisati kot modul za jedro, vendar sem kasneje 
ugotovil, da so funkcije, ki jih potrebujem za testiranje pravtako dostopne tudi v uporabniškem naèinu.\\
Testiral sem le L1 predpomnilnik, ker ima le ta harv in harvardsko arhitekturo, torej je loèen podatkovni in ukazni predpomnilnik.
Èe bi testiral L2 bi se zgodilo da bi se bloki ukazov in podatkov prekrivali, in nebi mogli izvesti toènih meritev.
\\
Za oblikovanje tega dokumenta je bil uporabljen sistem \LaTeX.

\vspace{1.3cm}
\noindent
{\large \bf Kljuène besede:}

\vspace{0.5cm}
\noindent
Predpomnilnik, asociativnost, pomnilniška hirearhija, testiranje.



%********************************************

\chapter{Predpomnilnik}

\section{Kaj je predpomnilnik}
Predpomnilnik je majhen a hiter pomnilnik, ki je navadno blizu centralne procesne enote. V novejših procesorjih je kar vgrajen v isto integrirano vezje, kot procesor. 
Namen predpomnilnika je zmanšanje dostopnih èasov do pomnilnika. V predpomnilniku so shranjene kopije delov glavnega pomnilnika, ki so pogosteje v uporabi. Vse skupaj 
sestavlja tako imenovano pomnilniško hirearhijo. Bolj ko je odstotek pomnilniških dostopov, ki so predpomnjeni visok, tem bolj se zakasnitev dostopa do pomnilnika  pribljižuje
zakasnitvi predpomnilnika. Ko centralna procesna enota potrebuje dostopati do glavnega pomnilnika, najprej preveri, èe je sluèajno na voljo kopija v predpomnilniku. Èe pride do
takoimenovanega zadetka v predpomnilniku procesor takoj dostopa do lokacije, ki je predpomnjena, kar je veliko hitrejše od dostopa do glavnega pomnilnika. Vse skupaj je mogoèe 
zato ker so na podlagi analiz programske opreme ugotovili, da pride do takoimenovane prostorske lokalnosti pomnilniških dostopov. Prostosrska lokalnost pomeni, da je velika verjetnost
da so dostopi, do pomnilniških lokaci tesno skupaj. Primeri so zanke in tabele.

\section{Vrste predpomnilnikov} 
Glede na naèin realizacije loèimo veè tipov predpomnilnikov. Pomnilniki so realizirani na veè naèinov zato, ker smo omejeni s ceno rešitvije in zaradi prostora, ki nam je na voljo
na silicijevi plošèici, kar posledièno vpliva na število tranzistorjev. Dejansko smo omejeni še zaradi drugih razlogov, kar pa presega obseg seminarske naloge. Bolj zahteven bralec
si dodatne razloge poišèe v \cite{Kodek2000}. Poznamo direktne, asociativne, setasociativne in psevdoasociativne predpomnilnike. 

\subsection{Direktni predpomnilnik}
Direktni predpomnilnik je najcenejši in najpreprostejši za realizacijo. Doloèen blok iz glavnega pomnilnika se lahko preslika v toèno doloèen
blok v predpomnilniku. Preslikav se naredi na podlagi spodnjih pomnilniških naslovov. Tako se pri manjših predpomnilnikih zgodi, pride do prekrivanja.
Do prekrivanja pride, ko želimo dostopati do naslova, ki ma enake spodnje pomnilniške naslove kot predpomnjena pomnilniška lokacija. Enake spodnje bite naslova pa imajo vsi naslovi,
ki so med seboj razmaknjeni za velikost predpomnilnika. Tukaj se že jasno vidi, da imamo problem pri dalših zankah, oziroma zankah, ki so daljše od velikosti predpomnlnika.
Tukaj je, kot dolžinska mera mišljeno število ukazov znotraj zanke. Vidimo, da lahko problem rešimo z poveèanjem velikosti predpomnika.

\subsection{Asociativni predpomnilnik}
Pri asociativnem ali èisttem asociativnem predpomnilniku, se doloèen blok iz glavnega predpomnilnika preslika v poljuben blok v predpomnilnik. Ker preslikava ni odvisna le od spodnjih
bitov naslova prekrivanja tukaj nimamo. Kaj kmalu pa se nam zgodi, da se nam predpomnilnik zapolni in potrebojemo odstraniti predpomnjen blok, da ga naredimo prostor za nov blok do katerega
trenutno dostopamo. Pri teh zamenjavah imamo veè razliènih strategij, najpogostejše sem opisal na stani \pageref{strategija_menjave}. 

\subsection{Set-asociativni predpomnilnik}
Set-asociativni predpomnilnik je nakakšna posplošitev direktnega in èistega asociativnega predpomnika. Ravno tako, kot pri direktnem predpomnilniku se bolok preslika glede na spodnje bite
dostopanega naslova, vendar pa si pri setasociativne lahko predstavljamo, da nam ti biti predstavljajo naslov èistega asociativnega pomnilnika v katerega naj se preslika. Primerjavo set-asociativnega in direktnega pomnilnika si lahko ogledate na sliki \ref{slika_primerjava}. Velikost enega 
od teh èistih asociativnih pomnilnikov imenujemo stopnja asociativnosti. Hitro lahko vidimo, da pri stopnji asociativnosti ena dobimo kar direktni pomnilnik. In obratno, pri stopnji asociativnosti,
ki je enaka številu blokov v predpomnilnikov, dobimo èisti asociativni predpomnilnik. Analogno, kot pri asociativnem pomnilniku prihaja znotraj teh majnih asociativnih predpomnilnikov do
zamenjav blokov, torej tudi tukaj se potrebujemo odloèati o zamenjevalni stategiji, katere sem opisal na stani \pageref{strategija_menjave}. Veèina današnjih procesorjev ima vgrajene prav setasociativne
predpomnilnike, zato ker predstavljajo nek kompromis med omejitvami in zmogljivostmi. Na sliki \ref{slika_missrate} imamo prikazano
odvisnost reciproèna vrednost delaža predpomnjenih blokov z velikostjo pomnilnika, za razliène vreste predopmnilnikov.

\begin{figure}[htb]
 \centering
 \includegraphics[width=13cm]{./cache_associative.png}
 % cache_associative.png: 546x271 pixel, 51dpi, 27.31x13.56 cm, bb=0 0 774 384
 \caption[Primerjava preslilikovanja]{Primerjava preslilikovanja pri direktnem in set-asociativnem predpomniku}
 \label{slika_primerjava}
\end{figure}

\subsection{Pseudo asociativni predpomnilnik}
Pri njem poteka dostop do pomnilnika podobno kot pri direktnem, èe imamo zadetek, ni nobene razlike. Ta se pojavi ob zgrešitvi. Namesto, da se v predpomnilnik prenese nov blok, se pri pseudo asociativnem naredi še en poizkus dostopa, tokrat do pseudo bloka.
Pseudo associativni predpomnilnik je podoben set asociativnemu z dvema blokoma v setu. Razlika je v tem, da je pri zadetku èas dostopa
do pseudo bloka daljši od èasa dostopa do prvega bloka.

\begin{figure}[htb]
 \centering
 \includegraphics[width=13cm,keepaspectratio=true]{./cache_missrate.pdf}
 % cache_missrate.svg: 400x320 pixel, 72dpi, 14.11x11.29 cm, bb=0 0 400 320
 \caption[Delež zgrešitev glede na velikost predpomnilnika]{Delež zgrešitev glede na velikost predpomnilnika na SPEC CPU2000.}
 \label{slika_missrate}
\end{figure}


\label{strategija_menjave}
\section{Strategije menjave bloka}
Ko je predpomnilnik polen, mora algoritem izbrati kateri blok naj zavrže da naredi prostor novemu bloku.\\
Odstotek zadetkov ali "hit rate" predpomnilnika opisuje kako pogosto je dostopan naslov dejansko predpomnjen. Uèinkovitejše menjalne
stategije beležijo informacio o uporabi bloka, da izbolšajo odstotek zadetkov.\\
Zakasnitev "latency" predpomnilnika opisuje koliko èasa potrebuje predpomnilnik, da prenese željeno pomnilniško lokacijo dejnasko v
procesor. Zaksnitev se poveèuje z kompleksnostjo strategije.\\
Vsaka menjalna strategija je kompromis med odstotkom zadetkov in zakasnitvijo.

\subsection{Belady-jev algoritem}
Je najbolj uèinkovit algoritem za menjavanje blokov, saj algoritem vedno zavrže blok, ki ne bo veè potrebovan najdalj èasa. Optimalni
rezultat je imenovan Belady-jev algoritem. Ker je v splošnem nemogoèe napovedati, kdaj bomo spet potrebovali doloèen blok, je algoritem nemogoèe implementirati, uporablja pa se za primerjavo uèinkovitosti drugih algoritmov.

\subsection{LRU algoritem}
Ta algoritem zavrže blok, ki ni bil v uporabi nadalj èasa. Algoritem potrebuje hraniti in posodabljati informacijo o tem, kateri blok
je bil kdaj uporabljen, kar pa je lahko èasovno potratno. Ponavadi implementacije tega algoritma zahtevajo tako imenovane starostne bite "age bits" za vsak blok "cache-line" v predpomnilniku. Pri takih implementacijah, vakiè ko dostopamo do doloèenega bloka
spremenimo starost vsem ostalim blokom. LRU je v bistvu družina menjalnih algoritmov.

\subsection{MRU algoritem}
Ta algoritem, v nasprotju z LRU, zavrže blok, ki je bil nazadnje dostopan. V knjigi \cite{VLDB1} "Ko je datoteka ponovlivo dostopna
[Zanke], je MRU najboljši menjalni algoritem". MRU algoritem je nadvse uporaben v situacijah kjer starejši je blok bolj verjetno bo
dostopan.

\subsection{Pseudo-LRU}
Za predpomnilnike z visoko stopnjo asociativnosti, navadno višjo od štiri, postanejo stroški implementaacije LRU algoritma previsoki.
Je strategija oziroma implementacija algoritma, ki skoraj vedno ugotovi blok, ki najdalj èasa ni bil dostopan, zadovoljiva.

\subsection{RR algoritem}
Algoritem nakljuèno izbere blok, ki naj bo zamenjan. Algoritem ne zahteve nobene informacije o zgodovini dostopov. Zaradi preprostosti
je uporabljen v ARM procesorjih.

\subsection{LFU algoritem}
Algoritem šteje kako pogosto dostopamo do nekega naslova. Tisti naslovi, ki so najmanjkrat dostopani, tiste bloke algoritem najprej zavrže.

\chapter{Metode in testiranje}

\section{Specifikacije raèunalnika} 
Testiral sem na prenosnem raèunalniku, procesor je nižjega cenovnega razreda. Samo specifikacijo procesorja lahko vidite na sliki \ref{slika_spec}. Slika je pridobljena z programskim orodjem CPU-Z. Katerega sem tudi uporabil za pridobitev prièakovanih rezultatov.

\begin{figure}[htb]
 \centering
 \includegraphics[width=8cm,keepaspectratio=true]{./CPU-Z_CPU.png}
 % CPU-Z_CPU.png: 409x394 pixel, 96dpi, 10.82x10.43 cm, bb=0 0 307 296
 \caption{Specifikacija procesorja}
 \label{slika_spec}
\end{figure}

\begin{table}[htb]
\begin{center}
\begin{tabular}{|l||l|}\hline
\textbf{Parameter}& \textbf{Vrednost parametra}\\\hline\hline
CPU	 & Intel M353\\\hline
Frekvenca& 900 Mhz\\\hline
Predpomnilnik L2& 512 kB\\\hline
Glavni pomnilnik & 2 GB\\\hline
Trdi disk & 160 GB\\\hline
\end{tabular}
\end{center}
\caption[Specifikacija raèunalnika.]{Specifikacija raèunalnika.}
\label{tabela_mere}
\end{table}

\section{Strategija testiranja} 
Najprej sem predpostavil, da je pomnilnik set asociativen brez kakšnih psevdo izvedb. Druga stvar, ki sem jo predpostavil pa je
da je menjalna strategija LRU.\\
Ob teh predpostavkah sem se odloèil, da bom dostopal do blokov sekvenèno po zaporednih naslovih, pri takih dostopih se LRU metoda
izkaže za zelo slabo, saj bi se morali menjati vsi bloki, v primeru, ko dostopamo do veè blokov, kot je stopnaja asociativnosti.
Tukaj seeveda mislimo bloke katerih se spodnji naslovi prekrivajo. V nasprotnem primeru, ko dostopamo do manj razliènih blokov, kot
je stopnaj asociativnosti, pa do menjave sploh ne prihaja. Tako imamo zgrešitev v predpomnilniku le na zaèetku, ko še ni nobenega bloka
v predpomniku. Razmerje teh zaèetnih zakasnitev pa gre z veèjim številom dostopov proti niè.

\section{Prièakovani rezultati} 
Prièakujemo torej, da za tiste teste, pri katerih je število razliènih blokov, do katerih dostopamo, manjše ali enako, stopnji asociativnosti, je povpreèen èas
dostopa kar pribljižno enak dostopu do L1. Izpeljavo lahko vidimo v spodnjih enaèbah.

\begin{table}
\begin{equation}
t_A = \lim{n \rightarrow \infinity}{ \frac{n*t_{L1} +  C*t_{L2}}{n}}            
\end{equation}\\
\begin{equation}
t_A = \lim_{n \rightarrow \inf}{ \frac{n*t_{L1}}{n} +  \frac{C*t_{L2}}{n}}         
\end{equation}\\
\begin{equation}
t_A = t_{L1} + 0            
\end{equation}\\
\begin{equation}
t_A = t_{L1}            
\end{equation}
\end{table}

V enaèbi lahko opazimo konstanto $C$. Ta konstanta nam pove do koliko razliènih blokov dostopamo.\\
Nasprotno, za tiste teeste, pri katerih je število razliènih blokov, do katerih dostopamo, veèje od stopnje asociativnosti, pa bi moral èas dostopa $T_A$
ustrezati spodnji enaèbi. Saj je potrebno ves èas menjavati bloke (Po naši predpostavki LRU).

\begin{table}
\begin{equation}
t_A = \lim{n \rightarrow \infinity}{ \frac{n*(t_{L1}+t_{L2})}{n}}            
\end{equation}\\
\begin{equation}
t_A = t_{L1}+t_{L2}           
\end{equation}
\end{table}


\chapter{Rezultati}

\section{Izpis programa}
Program poženemo za veè predpostavljenih stopenj. Pri predpostavljeni stopnji, ki je manjša oziroma enaka realni, prièakujemo podobne
dostopne èase. Ko pa presežemo mejo realne stopnje asociativnosti pa bi moral èas narasti. Izpis programa prikazuje slika \ref

\section{Razlaga}
Opazimo, da se dobljen rezultat razlikuje od prièakovanih. Razlog za to so seveda naše zaèetne predpostavke. Ker je raèunalnik nižjega
cenovnega razreda 


\subsection{Sklepne ugotovitve}

Sklepne ugotovitve naj prika?ejo oceno o opravljenem delu in povzamejo te?ave, na katere je naletel kandidat. Kot rezultat dela
lahko navede ideje, ki so nastale med delom, in bi lahko bile predmet novih raziskav.


\newpage

%********************************************

\appendix

%\addcontentsline{toc}{chapter}{\protect Dodatki}

\chapter{Koda uporabljena za testiranje}

\lstinputlisting[
language=C,
showspaces=false,
showstringspaces=false,
frame=lines,
keywordstyle=\color{blue},
commentstyle=\color{darkgreen},
stringstyle=\color{red},
breaklines=true,
breakatwhitespace=false,
title=\lstname,
basicstyle=\footnotesize\ttfamily\small,
emph={label},
morekeywords={asm}]
{testl1.c} 

\newpage

\addcontentsline{toc}{chapter}{Seznam slik}
\addtocontents{toc}{\protect\vspace{-2ex}}
\listoffigures

\newpage

\addcontentsline{toc}{chapter}{Seznam tabel}
\listoftables

%\listofalgorithms


%********************************************

\newpage

\addcontentsline{toc}{chapter}{Literatura}
\label{stran_literatura}

\begin{thebibliography}{99}

\bibitem{Kodek2000} D. Kodek, ``Predpomnilnik'' 
\textit{Arhitektura Raèunalniških Sistemov}, Ljubljana: BI-TIM, januar 2000, str. 314-346.

\bibitem{VLDB1} Hong-Tai Chou and David J. Dewitt, \textit{An Evaluation of Buffer Management Strategies for Relational Database Systems}, VLDB, 1985.
Dostopno na:\\
http://www.vldb.org/conf/1985/P127.PDF




\end{thebibliography}


\end{document}
