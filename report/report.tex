
\documentclass[12pt,a4paper,openany]{book}

%Uporabljeni paketi
\usepackage{fancyhdr}
\usepackage{graphicx}
\usepackage{color}
\usepackage{xcolor}
\usepackage[slovene]{babel}

\usepackage[utf8]{inputenc}
\usepackage[pdftex,bookmarks=true]{hyperref}

%Velikost strani - dvostransko
\oddsidemargin 1.4cm
\evensidemargin 0.35cm
\textwidth 14cm
\topmargin 0.26cm
\headheight 0.6cm
\headsep 1.5cm
\textheight 20cm

%Nastavitev glave in repa strani
\pagestyle{fancy}
\fancyhead{}
\renewcommand{\chaptermark}[1]{\markboth{\textsf{Poglavje \thechapter:\ #1}}{}}
\renewcommand{\sectionmark}[1]{\markright{\textsf{\thesection\  #1}}{}}
\fancyhead[RE]{\leftmark}
\fancyhead[LO]{\rightmark}
\fancyhead[LE,RO]{\thepage}
\fancyfoot{}
\renewcommand{\headrulewidth}{0.0pt}
\renewcommand{\footrulewidth}{0.0pt}

%********************************************

\begin{document}

% stran 1 med uvodnimi listi
\thispagestyle{empty} 

\begin{center}
{\large 
UNIVERZA V LJUBLJANI\\
FAKULTETA ZA RAČUNALNIŠTVO IN INFORMATIKO\\
}

 \includegraphics[scale=0.2,keepaspectratio=true]{./pictures/uni_logo.png}


\vspace{1.5cm}
{\LARGE Rok Kek, Matjaž Vrbole}\\

\vspace{2cm}
\textsc{\textbf{\LARGE 
Porazdeljene inteligentne programske tehnologije
}}

\vspace{2cm}
{ SEMINARSKA NALOGA}\\
{ POROČILO }\\

\vspace{2cm} 
{\Large Mentor: Luka Čehovin}

\vfill
{\Large Ljubljana, 2011}
\end{center}

\newpage

%********************************************

\renewcommand\thepage{} 
\tableofcontents 
\renewcommand\thepage{\arabic{page}}

\thispagestyle{empty}




\setcounter{page}{1}
\pagenumbering{arabic}

\chapter*{Povzetek}

\addcontentsline{toc}{chapter}{Povzetek}
Namen dokumenta je, predstaviti naše delo, ki se je izvajalo v sklopu vaj pri predmetu
 Porazdeljene inteligentne programske tehnologije. Cilj vaj je bil implementirati znanja
 in ideje, ki smo jih pridobili pri predmetu. Implementacija je v obliki agenta, ki sodeluje
 z drugimi agenti pri igranju igre Capture The Flag (CTF). Asistent pri predmetu nam je pripravil
 simulirano okolje agenta, v katerem se igra odvija.
\\
V tem dokumentu se bomo dotaknili problemov, kot so raziskovanje prostora z več agenti, sodelovanje med agenti, 
iskanje zastave, ter razne ofenzivne in defenzivne taktike pri sami igri.
\\
Pri raziskovanju prostora je cilj, podobno kot pri raziskovanju z enim samim agentom, minimizirati skupen porabljen čas.
 V našem primeru je cil, da agenti v čim krajšem času odkrijejo zastavo. Glavni problem, ki ga je potrebno rešiti v kontekstu
 raziskovanja z večimi agenti je izbira primernih smernic za posameznega agenta, tako da istočasno raziskujejo različne predele 
prostora. Tega smo se lotili tako, da vsak agent oceni ceno poti do ciljne točke in njen prispevek. Ob vsakem morebitnem srečanju 
pa si izmenjajo informacijo o prostoru, ki so jo do tistega trenutka pridobili.\\
\\
\\
Za oblikovanje tega dokumenta je bil uporabljen sistem \LaTeX.

\vspace{1.3cm}
\noindent
{\large \bf Ključne besede:}

\vspace{0.5cm}
\noindent
Agenti, sodelovanje, CTF, raziskovanje, algoritmi, AI.



%********************************************

\chapter{Predstavitev simuliranega okolja}
Okolje je implementirano v programskem jeziku Java. Okolje omogoča nalaganje različnih zemljevidov,
 postavitev zastave in izbiro načinov igranja. V sklopu vaj smo se osredotočili le na način pri katerem
 skupina agentov v čim krajšem času najde zastavo in jo prinese v domačo bazo. Na enem zemljevidu je tako
 več skupin agentov, zmaga tista skupina ki prinese zastavo prva.
\\
V simuliranem okolju imamo implementirano tudi senzoriko agenta, in akcije ki jih lahko izvaja.

\begin{figure}[ht]
 \centering
 \includegraphics[width=13cm]{./pictures/Simulacisko.png}
 \caption[Simulirano okolje]{Primer simuliranega okloja}
 \label{okolje}
\end{figure}

\section{Senzorji}
Senzorika agenta zajema prejemanje sporočil poslanih, od drugih agentov iste skupine in zaznavanja
prostora v katerem se giblje. V grafičnem vmesniku simuliranega okolja si lahko vidimo posredovanje
sporočil med agenti in pot ki jo je naredil vsak od agentov. 

\subsection{Prejemanje sporočil}
Agent sprejema sporočila, ki so mu bila poslana od drugih agentov. Sporočilo je v obliki znakovnega
zaporedja ali zaporedja bajtov. Je edini način komunikacije med agenti.

\subsection{Zaznavanje prostora}
Zaznavanje prostora je določeno z parametri okolja.  Agent zaznava kvadraten del prostora z središčem
njegovega položaja. Agent ločuje med različnimi objekti v prostoru, kot so; zid, prazen prostor, nasprotnik, nasprotnikova baza,
nasprotnikova zastava, njegova zastava, njegova baza in član njegove skupine z njegovo ID številko.


\section{Akcije agenta}
Agent lahko vpliva na okolje z svojim gibanjem, z pošiljanjem sporočil ter lahko nosi zastavo svoje ekipe.

\subsection{Gibanje}
Agent se lahko vedno premakne v katerokoli od štirih smeri (gor, dol, levo in desno), ne glede na to,
kaj je takrat v tisti celici. V primeru da se želi premakniti v celico, ki je označena kot zid, izgubi
življenje in se vrne v bazo, kjer zopet začne z novim preiskovanjem. Podobno se zgodi tudi takrat, če
se premakne v celico, kjer je že eden od njegovih ali nasprotnikovih robotov. V tem primeru življenje izgubita oba.

\subsection{Pošiljanje sporočil}
Pošiljanje sporočil je omejeno z razdaljo med agenti. Trenutne nastavitve okolja omogočajo posredovanje sporočil
 le agentom, ki so znotraj vidnega polja. Sporočilo se pošlje s pomočjo ID številke agenta ki je v isti ekipi.
 Dolžina sporočila je omejena, prav tako pa so sankcionirani agenti, ki med seboj prekomerno komunicirajo.

\subsection{Manipulacija z zastavo}
Agent lahko pobere zastavo svoje ekipe, in jo odnese v svojo bazo. Če med vračanjem domov agent umre zaradi
kakršnega koli vzroka, se zastava v zadnje obiskani celici spusti na tla in ostali agenti potem ko jo najdejo,
nadaljujejo od tam naprej. Igra dopušča tudi možnost, da je na enkrat na zemljevidu postavljenih več zastav za eno ekipo.



\chapter{Problemi na katere smo naleteli}

\section{Raziskovanje}
Program poženemo za več predpostavljenih stopenj. Pri predpostavljeni stopnji, ki je manjša oziroma enaka realni, pričakujemo podobne
dostopne čase. Ko pa presežemo mejo realne stopnje asociativnosti pa bi moral čas narasti. Izpis programa prikazuje slika 

\section{Iskanje poti do zastave}

V primeru, ko se zastava agentove skupine pojavi v njegovem vidnem polju ve katera je njegova ciljna celica,
ni pa nujno da obstaja pot po raziskanem prostoru, do te celice. V primeru da pot obstaja, je problem trivialen
in agent gere po zastavo. V primeru ko pa poti ni, pa je potrebno prostor nadaljne raziskati.\\
\\
V naši implementaciji agenta se s slednjim problemom soočamo tako, da agent sledi zidu v levo ali desno stran.
Tako so nekateri agenti levičarji in drugi desničarji, to se izračuna na podlagi njegove ID številke. Razdelitev
agentov na desničarje in levičarje nam reši dve stvari, to da agenti ne sledijo isti poti. Ter dopušča možnost,
da nek agent najde krajšo pot. Seveda vsem agentom, ki jih sreča na poti sporoči svoja spoznanja, tako da se lahko zgodi,
da manjkajoči del zemljevida izve od drugih agentov in problem postane zopet trivialen.


\section{Prenos zastave v domačo bazo}

Ko agent pobere zastavo je njegova naloga, da jo prenese v bazo svoje skupine.
 Pri tem pa lahko naleti na agenta nasprotne ekipe, ki ga po vsej verjetnosti 
poizkušal ovirati na njegovi poti. Agentu ne preostane drugega, kakor da se mu umika.
Vendar pa je gibanje agenta, ki nosi zastavo počasnejše od gibanja agenta, ki zastave nima.
Tako da je v tem primeru situacija neizbežna. Pomagamo si lahko le tako, da agenti iste 
skupine ščitijo agenta z zastavo.\\
\\
Agenta lahko ščitimo na dva načina. Prvi način je ta, da se z agentom, ki ne nosi zastave,
poizkušamo zaleteti v  agenta nasprotne skupine. Tako ima naš agent z zastavo prosto pot. 
Drugi način pa je ta, da agent brez zastave le spremlja agenta, ki zastave nima. Ko se 
nasprotnikov agent zaleti v agenta z zastavo (ob tem seveda oba agenta izgubita življenje), 
tako lahko preživeli agent pobere zastavo in nadaljuje pot proti bazi. Odločili smo se za 
implementacijo drugega primera ščitenja agentov, saj je lažji in prav tako učinkovit.


\section{Srečanje z drugimi agenti}

Pri srečanju agenta iz iste ekipe, ob seveda prednosti komunikacije in posredovanja  medsebojnih informacij,
obstaja nevarnost medsebojnega trka. To predstavlja nevarnost, da dva agenta izgubita življenje dva agenta.\\
\\
Ta problem smo rešili tako, da se agenti z nižjimi ID številkami ne ozirajo na agente z višjimi ID, jim pa 
sporočijo svojo namero gibanja. Tako se lahko agenti z višjo ID prilagodi agentu z nižjo. Izjema je primer, 
ko agent nosi zastavo, takrat ima ta agent prednost pred vsemi. Kako drugi agenti reagirajo na konfliktno 
stanje je odvisno od njihovega notranjega stanja. Če iščejo zastavo ali raziskuje prostor potem, najdejo pot 
mimo celice kamor se odpravlja prioritizirani agent. Če dobi sporočilo da agent nosi zastavo, potem mu začne 
slediti, in ga na ta način varuje pred sovražniki.

\subsection{Srečanje z agenti nasprotne ekipe}
Če se agent sreča z agentom nasprotne skupine pomeni, da je v njegovem vidnem polju.
V primeru, da je kateri izmed teh dveh agentov “nepazljiv” in se premakne v celico, 
kjer je drug agent, lahko pride do medsebojnega trka. Če agentova strategija ni usmerjena k
uničevanju drugih agentov in s tem posledično tudi samega sebe, je edina logična pot ta, da
se agenta poskušata zaobiti in s tem ohranita možnost iskanja vsak svoje zastave.\\
\\
Naši agenti so sprogramirani tako, da se poskušajo izogibati agentom nasprotne skupine. 
Ko se nasprotni agent približa našemu agentu na 1 celico narazen, se naš agent premakne 
v nasprotno smer od sovražnega agenta. To pomeni, če naš agent vidi nasprotnika levo od 
svoje trenutne pozicije, se bo v naslenjem koraku premaknil desno, če tam seveda ni kakšne 
druge ovire (stena, agent). Izjema je primer ko je agent varovan oziroma ima spremstvo, torej
nese zastavo proti bazi. V tem primeru, pa agent ne reagira na druge agente, temveč nese zastavo
naravnost proti bazi.
\newpage

%********************************************


\addcontentsline{toc}{chapter}{Seznam slik}
\addtocontents{toc}{\protect\vspace{-2ex}}
\listoffigures

\newpage



%********************************************

\newpage

\addcontentsline{toc}{chapter}{Literatura}
\label{stran_literatura}

\begin{thebibliography}{99}

\bibitem{Kodek2000} D. Kodek, ``Predpomnilnik'' 
\textit{Arhitektura Računalniških Sistemov}, Ljubljana: BI-TIM, januar 2000, str. 314-346.

\bibitem{VLDB1} Hong-Tai Chou and David J. Dewitt, \textit{An Evaluation of Buffer Management Strategies for Relational Database Systems}, VLDB, 1985.
Dostopno na:\\
http://www.vldb.org/conf/1985/P127.PDF




\end{thebibliography}


\end{document}

